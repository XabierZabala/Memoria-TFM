% Tipo de documento
\documentclass[12pt, twoside, a4paper, openany, bibliography=totoc, numbers=noenddot]{scrbook}

% Añadir caracteres no anglosajones como tildes, ñ, ¿, ¡, etc.
\usepackage[utf8]{inputenc}
\usepackage[spanish]{babel}
 
% Añadir gráficos
\usepackage{graphicx}
% Carpeta donde se encuentran las imágenes
\graphicspath{ {figs/} }
\usepackage[labelfont=bf]{caption}
\usepackage{subcaption}
\DeclareGraphicsExtensions{.pdf,.png,.jpg}
\usepackage{chngcntr}
\counterwithout{figure}{chapter}

% Listas con corchetes tipo [1], [2]...
\usepackage{enumitem}

% Permite usar hipervínculos
\usepackage[hidelinks]{hyperref}

\usepackage{afterpage}

\newcommand\blankpage{
    \null
    \thispagestyle{empty}
    \newpage}

% Floating options
\usepackage{float}
\restylefloat{figure}

% Márgenes
\usepackage[left=2.5cm,right=2.5cm,bindingoffset=0.5cm]{geometry}
\setlength{\headheight}{20pt}

% Estilo de los titulos de los capítulos
\usepackage{titlesec}
\titleformat{\chapter}[display]
{\normalfont\huge\bfseries}{\chaptertitlename\ \thechapter}{20pt}{\Huge}[\vspace{2ex}\titlerule]

% Fuente utilizada para el cuerpo
\usepackage[bitstream-charter]{mathdesign}

% Permite usar frames (cajas)
\usepackage{framed}

% Permite usar colores
\usepackage[usenames,dvipsnames,svgnames,table]{xcolor}

% Permite el uso de cabeceras y pies de página
\usepackage{fancyhdr}

% Biblatex
\usepackage[backend=bibtex,style=numeric,natbib=true]{biblatex}
\addbibresource{bibliography.bib}

% Permite realizar rotaciones
\usepackage{rotating}

% Opciones de tablas
% Para crear líneas más gruesas
\usepackage{tabu}
\counterwithout{table}{chapter}

\captionsetup[figure]{font=bf,position=below}

% Prevents placing floats before the section 
\usepackage{placeins}
\makeatletter
\AtBeginDocument{%
  \expandafter\renewcommand\expandafter\subsection\expandafter
    {\expandafter\@fb@secFB\subsection}%
  \newcommand\@fb@secFB{\FloatBarrier
    \gdef\@fb@afterHHook{\@fb@topbarrier \gdef\@fb@afterHHook{}}}%
  \g@addto@macro\@afterheading{\@fb@afterHHook}%
  \gdef\@fb@afterHHook{}%
}
\makeatother

\PassOptionsToPackage{usenames,dvipsnames}{xcolor}

\usepackage[usenames,dvipsnames]{xcolor}
\usepackage[draft]{pgf}
\usepackage{listings}
\usepackage[svgnames]{xcolor}

% Bordes en imágenes
\usepackage[export]{adjustbox}

% Múltiples líneas en una misma celda de una tabla => \specialcell{}
\newcommand{\specialcell}[2][c]{%
  \begin{tabular}[#1]{@{}c@{}}#2\end{tabular}}

\lstset{
     language        = php,
     basicstyle      = \small\ttfamily,
     keywordstyle    = \color{blue},
     stringstyle     = \color{red},
     identifierstyle = \color{ForestGreen},
     commentstyle    = \color{gray},
     emph            =[1]{php},
     emphstyle       =[1]\color{black},
     emph            =[2]{if,and,or,else},
     showstringspaces=false,
     emphstyle       =[2]\color{yellow},
     backgroundcolor=\color{gray!10},
     breaklines=true,
     numbers=left,
     numberstyle=\footnotesize,
     showspaces = false,
     showstringspaces = false,
     tabsize = 2,
     %numbers=left,
     %numbersyle=\tiny
     frame=single,
     xleftmargin=5pt,
     xrightmargin=3pt,
     aboveskip = 20pt,
     rulecolor=\color{black},
     escapechar=|
}
     
\renewcommand{\lstlistingname}{Código}
\DeclareCaptionFormat{listing}{\rule{\dimexpr\textwidth\relax}{0.4pt}\par\vskip1pt#1#2#3}
\captionsetup[lstlisting]{format=listing,singlelinecheck=false, margin=0pt, font={sf},labelsep=space,labelfont=bf}

\begin{document}

\fancyhead[R]{\slshape \rightmark}
\fancyfoot[C]{\thepage}

% Permite escoger la profundidad de las secciones (1.1, 1.1.1.2...)
\setcounter{secnumdepth}{2}

%----------------------------------------------------%
%                      PORTADA                       %
%----------------------------------------------------%

\pagestyle{empty}

% Define una línea horizontal para el título
\newcommand{\HRule}{\rule{\linewidth}{0.5mm}} 

% Centra el contenido
\begin{center}
	% Título entre dos líneas horizontales
	\HRule \\[0.5cm]
	\vspace{0.5cm}
	\textbf {
		{\huge título}\\
		\vspace{0.3 cm}
		Procesamiento masivo de datos mediante Cassandra y Spark\\
	}
	\vspace{0.5cm}
	\HRule \\[0.5cm]
	{\large
		
		\vspace{1 cm}
		Máster en Sistemas Informáticos Avanzados\\
		Septiembre de 2016\\
		\vspace{3.0 cm}
		Autor:\\
		\vspace{0.2 cm}
		Xabier Zabala Barandiaran\\
		\vspace{1.0 cm}
		Supervisores:\\
		\vspace{0.2 cm}
		German Rigau i Claramunt\\
		{\small UPV/EHU\\}
		Iñigo Etxabe\\
		{\small Datik Información Inteligente S.L.\\}
	}

	\vspace{2.0 cm} 
	\begin{figure}[h!]
		\centering
		\includegraphics[width=0.4\textwidth]{Ilustraciones/ehu.png}\hfill
		\includegraphics[width=0.4\textwidth]{Ilustraciones/informatica.png}\hfill

	\end{figure}
\end{center}
\frontmatter
\pagestyle{plain}
\cleardoublepage

%----------------------------------------------------%
%                  AGRADECIMIENTOS                   %
%----------------------------------------------------%

\begin{flushright}
	\Large\textit{Agradecimientos}
\end{flushright}

En primer lugar, quisiera expresar mi gratitud a las personas que han posibilitado la concepción y el desarrollo de la presente tesina. Agradezco a German Rigau i Claramunt, director del proyecto por parte de la UPV/EHU, el asesoramiento ofrecido durante el transcurso del mismo. Doy también las gracias a Iñigo Etxabe, cofundador y CTO de Datik Información Inteligente, por la total confianza mostrada en mi labor y el exquisito trato ofrecido desde el primer día.\\

Agradecer, cómo no, a mis padres José Javier Zabala y María Pilar Barandiaran el esfuerzo desempeñado para allanar, en la medida que les ha sido posible, el camino que he recorrido hasta el día de hoy. Me congratula haber sabido responder satisfactoriamente a la confianza que han depositado en mí. Por todo lo que suponen para mí, un beso enorme para los dos.\\

No quisiera olvidarme de aquellas personas que me han acompañado durante este maravilloso periplo. Doy las gracias a los amigos de siempre por el apoyo incondicional mostrado desde la distancia y a la gente que he tenido la fortuna de conocer durante los años de universidad. Quisiera agradecer especialmente a las personas que en este breve periodo se han ganado a pulso el privilegio a ser parte importante de lo que me resta de existencia, a los cuales me atrevo a mencionar aún con el temor de dejarme a alguna en el tintero: Adrián Núñez, Eider Irigoyen, Jaime Altuna, Jokin Etxeberria, Marta García, Mikel Etxeberria y Uxue Arostegui. No hallo palabras para expresar la gratitud que siento hacia vosotras.\\

Por último, pero no por ello menos importantes, quisiera evocar a todos los docentes que han tomado parte en mi formación desde aquel Septiembre del 2009 y agradecerles el esfuerzo invertido en mi persona durante estos años.\\

Gracias de todo corazón a la gente mencionada en este breve capítulo por haber hecho de mí un mejor profesional y una mejor persona, además de darme las fuerzas necesarias para seguir evolucionando en ambos aspectos de cara al futuro.


\cleardoublepage

%----------------------------------------------------%
%                      RESUMEN                       %
%----------------------------------------------------%

% Castellano
%\section*{Resumen}

Proyecto Final del Máster en Sistemas Informáticos Avanzados. Estudio realizado sobre las tecnologías emergentes en el campo del Big Data y los beneficios que podrían aportar en escenarios con problemas cada vez más latentes en cuanto a almacenamiento y procesamiento de datos debido al inexorable aumento del volumen.

%%%%%%%%%%%%%



Proyecto de Fin de Grado de la especialidad de Computación. Se ha implementado la aplicación para el seguimiento de ejercicios en el aula llamada \textbf{exerClick}. Es una aplicación multiplataforma para móviles, evaluada en Android y en iOS y adaptada a esos sistemas gracias a la plataforma Cordova \cite{apachecordova}. En su implementación se han utilizado las tecnologías web HTML5, CSS3 y Javascript, además de PHP para el servidor.\\

La aplicación permite que los profesores añadan ejercicios y los alumnos le envíen \textit{feedback} sobre su realización mediante dos opciones: marcar una duda en el ejercicio o marcar el ejercicio como finalizado.\\

%%%%%%%%%%%%%
En el presente trabajo de fin de máster desarrollado en la empresa Datik se ha abordado la problemática del procesamiento masivo de datos con el objetivo de dilucidar si es posible hacer frente a este reto con los medio disponibles en la empresa.

Tras analizar las diferentes tecnologías existentes y los recursos disponibles en la empresa se ha construido un cluster altamente escalable basado en máquinas virtuales. Gracias a Apache Cassandra y Apache Spark, dicho cluster es capaz de paralelizar el computo repartiendo las tareas entre los diferentes nodos que lo componen. 

Varios test han sido diseñados y ejecutados tanto en el cluster como en una base de datos MySQL con la intención de verificar el rendimiento ofrecido por el primero. Siendo los resultados obtenidos desfavorables para la infraestructura propuesta, se ha analizado las causas de este hecho. 

Como resultado del análisis, se puede concluir que las tecnologías Apache Cassandra y Apache Spark cumplen con sus propósitos pero los recursos que dispone Datik, a día de hoy, no son los adecuados para sacar partido de estos. Realizando una inversión moderada sería factible apostar por estas tecnologías para lo que se recomienda invertir en contratar un servicio IaaS.

Keywords: ITS, Apache Spark, Apache Cassandra, procesamiento distribuido, Big Data.

%\cleardoublepage

%----------------------------------------------------%
%                    INDICE GENERAL                  %
%----------------------------------------------------%

%\tableofcontents
%\newpage

%----------------------------------------------------%
%                    INDICE FIGURAS                  %
%----------------------------------------------------%

%\listoffigures
%\newpage

%----------------------------------------------------%
%                    INDICE TABLAS                   %
%----------------------------------------------------%

%\listoftables
%\cleardoublepage

%----------------------------------------------------%
%                    INTRODUCCION                    %
%----------------------------------------------------%

%\mainmatter
%\fancyhead[LE,RO]{\itshape \nouppercase \rightmark}
%\fancyhead[LO,RE]{\itshape \nouppercase Capítulo \arabic{chapter}}

%%----------------------------------------------------%
%                    INTRODUCCION                    %
%----------------------------------------------------%

\pagestyle{fancy}

\chapter{Introducción}
\label{introduccion}

Desde Aristóteles y su libro Segundos Analíticos \footnote{\href{https://docs.google.com/a/datik.es/file/d/0By4kcbi6MzzdUHhVQnUtcTNUdk0/view}{Órganon II de Aristóteles: Segundo Analíticos se encuentra recopilado en él}} hasta Galileo, padre de la ciencia moderna, adalides del conocimiento han proclamado que un método de investigación basado en lo empírico y en la medición, sujeto a los principios específicos de las pruebas de razonamiento es el camino para conocer la verdad.\\

Hoy en día, época en la que los avances tecnológico han posibilitado observar y medir de forma exhaustiva un gran abanico de fenómenos, la ingente cantidad de datos que se genera en el proceso es, a veces, intratable para las tecnologías convencionales, y por ende, es imposible extraer todo el conocimiento que atesoran. El problema, lejos de atenuarse, se acrecienta con el paso del tiempo. Estudios como el realizado por McKinsey Global Institute estiman que el volumen de datos que se genera crece un 40\% cada año y auguran que entre 2009 y 2020 se verá multiplicado por 44 \cite{nambiartowards}.\\

Por ello, en los últimos años ha irrumpido la necesidad de hallar nuevas metodologías y tecnologías que permitan procesar y extraer el conocimiento que atesora el torrente de información en la cual se encuentra envuelta la sociedad, dando como resultado el nacimiento del Big Data.\\

El mundo empresarial, por su parte, no se ha mantenido al margen de esta gran revolución. Conscientes de los beneficios que les puede reportar en diferentes aspectos de su negocio, la gran mayoría de las empresas se han interesado en este fenómeno. De un estudio realizado entre los altos ejecutivos de las firmas que lideran el Wall Street se desprende que el 96\% tiene planeadas ciertas iniciativas relacionadas con el Big Data, y el 80\% ya tiene finalizada alguna \cite{bdes:2013}. 

\section{Contexto}
 
Datik Información Inteligente \footnote{\url{http://www.datik.es/}} es una empresa tecnológica perteneciente al Grupo Irizar \footnote{\url{http://www.irizar.com/irizar/}}  que desarrolla soluciones ITS destinadas a la gestión del trasporte, tanto ferroviario como por carretera y movilidad ciudadana.\\

Uno de los productos estrella de la entidad es el denominado iPanel, concentrador de  información que ofrece al operador de transporte servicios de valor añadido en la gestión de la información generada por su flota. El funcionamiento del servicio se puede resumir mediante la Figura \ref{fig:ipanel}:\\

\begin{figure}[h]
	\centering
	\includegraphics[width=1\textwidth]{Ilustraciones/ipanel_infraesctructure.png}
	\caption{Funcionamiento resumido de iPanel}
	\label{fig:ipanel}
\end{figure}

La incesante integración de nuevos vehículos a iPanel ha generado un crecimiento exponencial en el número de registros almacenados en ciertas tablas MySQL. Aunque el volumen actual no suponga riesgo alguno para el funcionamiento del servicio, Datik tiene identificados varios escenarios en los que la situación se podría revertir, causando graves problemas en el sistema.\\

Un problema, intrínseco a depender de una base de datos centralizada, es el operar sobre un único punto de fallo. Debido a que la mayoría de procesos confluyen en ella, el bloqueo o la caída causada por un servicio puede acarrear la de otros, a priori, totalmente independientes. Para solventar el problema Datik ha optado por migrar su infraestructura a una basada en Microservicios \cite{newman2015building}, logrando de esa manera, el aislamiento total de los componentes que conforman su ecosistema.\\ 

Otro problema, es el referente al proceso denominado Cálculo de Indicadores, el cual se ejecuta una vez al día para realizar operaciones aritméticas sobre diversas tablas y después, agrupar los resultados en base a diferentes criterios. Siendo dichas tablas las que mayor crecimiento experimentan, el aumento del volumen de las mismas incrementa de forma desorbitada el tiempo necesario para finalizar el cálculo, pudiendo, en un futuro, llegar a tardar mas de 24 horas y cancelar los indicadores que ofrecen información del último día.\\

El objetivo del presente proyecto es proponer soluciones a los problemas planteados en este apartado, prestando especial atención al proceso Cálculo de Indicadores, ya que afecta de forma directa a la información que consumen los clientes de Datik vía iPanel.\\

\section{Propuesta}

La problemática que envuelve al Cálculo de Indicadores es originado por el incremento exponencial de datos que dicho proceso ha de tratar. No sería descabellado pensar que la solución pueda pasar por escalar la máquina verticalmente y afinar la configuración de MySQL. No obstante, ambas mejoras tienen un límite, mientras que el volumen de los datos seguirá aumentando de forma inexorable, volviendo, tarde o temprano, a tener que lidiar con los mismos problemas del principio.\\

Siendo imposible reconducir la situación usando las tecnologías tradicionales, en el presente proyecto se propone realizar un cambio de paradigma que implica migrar las tablas relacionadas con los indicadores a un modelo distribuido que posibilite escalar la infraestructura horizontalmente. A su vez, se sugiere dividir el problema en dos apartados, almacenamiento y procesado, dotando la infraestructura de tecnología adecuada para ofrecer una respuesta eficaz a cada una de las partes.\\

Los datos consultados por el proceso Cálculo de Indicadores no presentan relación alguna con el resto de las entidades del esquema, por ello, se propone utilizar una base de datos no relacional para almacenarlos. Dentro de la extensa gama de sistemas de almacenamiento NoSQL que existen hoy en día, se ha considerado que Apache Cassandra \cite{lakshman2010cassandra} es el idóneo para solventar el problema. Se trata de una base de datos distribuida de alta disponibilidad que opera sin un único punto de fallo gracias a la replicación asíncrona peer-to-peer. Lo que le diferencia de sus homólogos que, además, se diferencia del resto por ser el que mejores ratios ecrituras/seg ofrece ha sido optimizada para  escrituras lo cual posibilita  (caracteristoca que viene como anillo al dedo)



Analizando el escenario presentado, 

** Los datos con los cuales se hara a prueba no son de datik, por motivos de confidencialidad de cantidad

Apache Cassandra es una base de datos distribuida no-sql. Gracias a naturaleza distribuida ayuda a resolver, 

Funcionalidades nuevas que trabajan con videos etc



(testear las tecnologias con dataset publico, los datos de la empresa son confidenciales) Debido a la falta de datos se ha utilizado un dataset publico para emular las condiciones de futuro con las que se va a encontrar datik

\section{Organización del documento}

En esta memoria se ha documentado  el desarrollo de la herramienta \textbf{\textit{exerClick}}, dentro del Trabajo de Fin de Grado (TFG) del autor. En el documento se describe la propuesta, la planificación y gestión que esta lleva consigo, la implementación llevada a cabo y las conclusiones finales.\\

En este primer capítulo se ha introducido el problema a resolver y se ha explicado la propuesta presentada en este proyecto.\\

En el capítulo 2 se presenta el Documento de Objetivos de Proyecto (DOP). Este recoge el alcance y las fases y tareas del proyecto, el análisis de riesgos y el análisis de factibilidad.\\

Una vez en el capítulo 3 se explica la gestión llevada a cabo durante el proyecto. Se presentan las metodologías utilizadas: Metodologías Ágiles e InterMod (adaptada a las necesidades de este proyecto). A continuación se detallan cada una de las iteraciones llevadas a cabo (como parte de la metodología InterMod): duración, objetivos y tareas realizadas. Al final del capítulo se muestra la documentación asociada a las iteraciones y los objetivos, además del seguimiento de tiempo realizado.\\

A continuación, en el capítulo 4 se detalla el análisis de requisitos. Primero se detallan los requisitos no-funcionales y luego los funcionales (prototipos en papel llevados a cabo durante las primeras iteraciones que dan una visión global del proyecto).\\

En el capítulo 5 se explica el diseño e implementación llevados a cabo. Se comienza mostrando la estructura de documentos del proyecto, luego el diseño realizado en base al análisis de requisitos del capítulo 4 y finalmente una visión general de la implementación de la lógica de negocio.\\

Para finalizar, en el capítulo 6 se presentan las conclusiones, líneas futuras para el proyecto y las lecciones aprendidas.\\

Fuera de la estructura general de la memoria, tenemos la bibliografia y los apéndices. En estos últimos tenemos las actas de reuniones, las actas de pruebas y la vista de relaciones de la base de datos (de la parte utilizada o creada específicamente para el proyecto).\\

%----------------------------------------------------%
%                     OBJETIVOS                      %
%----------------------------------------------------%

%\cleardoublepage
%\input{capitulos/2-objetivos.tex}

%----------------------------------------------------%
%                GESTIÓN DEL PROYECTO                %
%----------------------------------------------------%

%\cleardoublepage
%\input{capitulos/3-gestion.tex}

%----------------------------------------------------%
%              ANALISIS DE REQUISITOS                %
%----------------------------------------------------%

%\cleardoublepage
%\input{capitulos/4-analisis-de-requisitos.tex}

%----------------------------------------------------%
%              DISEÑO E IMPLEMENTACIÓN               %
%----------------------------------------------------%

%\cleardoublepage
%\input{capitulos/5-diseno-e-implementacion.tex}

%----------------------------------------------------%
%                     CONCLUSIONES                   %
%----------------------------------------------------%

%\cleardoublepage
%%----------------------------------------------------%
%                     CONCLUSIONES                   %
%----------------------------------------------------%

\pagestyle{fancy}

\chapter{Conclusiones y trabajos futuros}
\label{conclusiones}



\section{Conclusiones}

Tal y como se ha podido apreciar durante el transcurso de la presente tesina, muchas son las ventajas ofrecidas por las tecnologías Apache Cassandra y Apache Spark. Entre ellas, cabe resaltar la posibilidad de operar sin punto único de fallo y escalar tanto en almacenamiento como procesamiento superando, además, inequívocamente, los tiempos de respuesta ofrecidos por MySQL.\\

Entramado

por un lado el cambio de paradigma que supone

-Añadir la experiencia adquirida durante este ultimo año y medio trabajando en producción

- Dificil de crear nuevas consultas por que necesita un gran desarrollo, en entornos de constante evolucion pues malamente...

-Cassandra y Spark son tecnologias muy jovenes con un futuro enorme,constante evolucion..., pero en produccion... lo cual hacen que la implantación sea lenta y lleno de baches

- se necesita formar a mucho personal

\section{Trabajos futuros}

//
Calcular los tiempos de insercion correctos aplicando los cambios pertinentes en el disco (ha desvirtuado los resultados obtenidos lo cual es una pena)

// algo sobre spark

//
Se considera de especial interes... 
probar el cluster de mysql añadiendo las restricciones que tiene cassandra y comparar el resultado ofrecido por ambos
40 años a sus espalda tiene mysql, 
igual para una empresa como datik ofrece mas seguridad y estabilidad esta otra solucion (por probar que no quede)



%----------------------------------------------------%
%                    BIBLIOGRAFIA                    %
%----------------------------------------------------%

%\cleardoublepage
%\nocite{*}
%\printbibliography[heading=bibintoc,title={Bibliografía y Referencias}]

%----------------------------------------------------%
%                    APENDICES                       %
%----------------------------------------------------%

% Reiniciamos el contador de capítulos y hacemos que Capítulo pase a ser Apédice
%\setcounter{chapter}{0}
%\renewcommand{\chaptername}{Apéndice } % dejar el espacio es importante
%\renewcommand{\thechapter}{\Alph{chapter}}

% Cambiamos el titulo para que ponga Apéndice en lugar de Capítulo tal y como acabamos de definir
%\titleformat{\chapter}[display]
%{\normalfont\huge\bfseries}{\chaptername \thechapter}{20pt}{\Huge}[\vspace{2ex}\titlerule]

%\fancyhead[LE,RO]{\itshape \nouppercase \rightmark}
%\fancyhead[LO,RE]{\itshape \nouppercase \chaptername \Alph{chapter}}

%\cleardoublepage
%\input{apendices/a-actas-reuniones.tex}

%\cleardoublepage
%\input{apendices/b-actas-pruebas.tex}

%\cleardoublepage
%\input{apendices/c-base-de-datos.tex}

%\backmatter

\end{document}