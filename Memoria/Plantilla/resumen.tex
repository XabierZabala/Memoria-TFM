\section*{Resumen}

Proyecto Final del Máster en Sistemas Informáticos Avanzados. Estudio empírico sobre el rendimiento ofrecido por varias tecnologías emergentes en el campo del Big Data en comparación a una base de datos tradicional a la hora de operar en escenarios que requieren un almacenamiento y procesamiento eficaz de volúmenes masivos de datos.\\

Para llevar a cabo el experimento, dos entornos de prueba totalmente aislados han sido erigidos sobre la misma máquina física. En el primero, se ha construido un clúster compuesto por tres nodos virtuales que operan sobre una misma red privada. Dichos nodos han sido dotados de tecnología necesaria para el correcto funcionamiento de Apache Cassandra \cite{lakshman2010cassandra} y Apache Spark \cite{zaharia2010spark}. En el segundo entorno, se ha instalado MySQL Server sobre un nodo virtual de potencia equivalente al clúster mencionado con anterioridad. Una vez habiendo poblado las bases de datos mediante un data-set público de aproximadamente 25GB y diseñadas unas consultas acordes a la naturaleza de los datos, se han ejecutado dichas consultas para así cuantificar el tiempo de respuesta en cada escenario.\\

El estudio evidencia que a la hora de trabajar con volúmenes masivos de datos el binomio entre Apache Cassandra y Apache Spark mejora sustancialmente los tiempos de procesamiento obtenidos con MySQL, además de ofrecer una solución totalmente escalable y tolerante a fallos. No obstante, para gozar de dichas ventajas  se antoja necesario un análisis previo de los datos que se desean tratar, aspecto en el que MySQL demuestra ofrecer una mayor libertad.\\


Palabras Clave: Apache Cassandra, Apache Spark, Big Data, Clustering, Comparativa, NoSQL.\\
