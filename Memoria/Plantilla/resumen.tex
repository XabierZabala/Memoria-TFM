\section*{Resumen}

Proyecto Final del Máster en Sistemas Informáticos Avanzados. Estudio comparativo de carácter empírico realizado sobre el rendimiento ofrecido por varias tecnologías emergentes en el área del Big Data a la hora de operar en escenarios que requieren un almacenamiento y procesamiento eficaz de volúmenes masivos de datos.\\

Se ha construido un clúster compuesto por tres nodos virtuales que operan sobre Linux. Una vez configurados y dotados de tecnología necesaria para el correcto funcionamiento de MySQL Cluster \cite{mysqlcluster} y Apache Cassandra \cite{apachecassandra}, se ha procedido a poblar dichos sistemas de almacenamiento utilizando un data-set público de aproximadamente 25GB. Un conjunto de consultas diseñadas atendiendo a la naturaleza de los datos almacenados han sido ejecutadas sobre ambas infraestructuras. El uso de Apache Spark \cite{apachespark} durante el proceso ha posibilitado la distribución de la carga computacional entre los nodos que componen el clúster. A su vez, se ha erigido un nodo virtual con especificaciones técnicas equivalentes a la suma de los 3 nodos virtuales que forman el clúster para emular la respuesta ofrecida por una base de datos MySQL tradicional en el mismo contexto.\\

El estudio evidencia que, entre las tecnologías de almacenamiento distribuido comparadas, Apache Cassandra es la alternativa más eficaz a la hora de tratar volúmenes masivos de datos a costa de tener que realizar un análisis y diseño previo de los mismos. No obstante, MySQL Cluster es una opción muy a tener en cuenta, ya que sacrificando la eficacia de forma ligera, posibilita estructurar datos de manera más flexible. Ambas tecnologías demuestran estar mejor capacitadas que el MySQL monolítico tradicional para operar en el contexto analizado.\\


Palabras Clave: Big Data, MySQL Cluster, Apache Cassandra, Apache Spark, Clústering.\\
