\section*{Resumen}

Proyecto Final del Máster en Sistemas Informáticos Avanzados. Estudio empírico sobre el rendimiento ofrecido por varias tecnologías emergentes en el campo del Big Data en comparación a una base de datos tradicional a la hora de operar en escenarios que requieren un almacenamiento y procesamiento eficaz de volúmenes masivos de datos.\\

Dos entornos de pruebas totalmente aislados han sido erigidos sobre la misma máquina física. En el primero, se ha construido un clúster compuesto por tres nodos virtuales que operan en la misma red privada. Dichos nodos han sido dotados de tecnología necesaria para el correcto funcionamiento de Apache Cassandra \cite{lakshman2010cassandra} y Apache Spark \cite{zaharia2010spark}. Sobre el segundo entorno, constituido por un nodo virtual de potencia equivalente al clúster ya mencionado, se ha instalado una instancia de MySQL Server. Una vez habiendo diseñado un conjunto de consultas equivalentes para ambas bases de datos, se ha procedido a poblar dichos sistemas de almacenamiento utilizando un data-set público de aproximadamente 25GB. Para finalizar, las consultas predefinidas han sido ejecutadas para así poder cuantificar el tiempo de respuesta ofrecido por cada entorno.\\

El estudio evidencia que a la hora de trabajar con volúmenes masivos de datos el binomio formado por Apache Cassandra y Apache Spark, además de ofrecer una solución totalmente escalable y tolerante a fallos, mejora sustancialmente el rendimiento ofrecido por MySQL Server. No obstante, para gozar de dichas ventajas, se antoja necesario invertir más tiempo en analizar exhaustivamente la naturaleza de los datos que se desean tratar.\\


Palabras Clave: Apache Cassandra, Apache Spark, Benchmark, Big Data, MySQL Server.\\
