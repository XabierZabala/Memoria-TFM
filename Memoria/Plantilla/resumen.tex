\section*{Resumen}

Proyecto Final del Máster en Sistemas Informáticos Avanzados. Estudio comparativo de carácter empírico realizado sobre el rendimiento ofrecido por varias tecnologías emergentes en el área del Big Data a la hora de operar en escenarios que requieren un almacenamiento y procesamiento eficaz de volúmenes masivos de datos.\\

Se ha construido un clúster compuesto por tres nodos virtuales y en él se han instalado y configurado  MySQL Cluster  las bases de datos distribuidas MySQL Cluster y Apache Cassandra, objetos de análisis en el presente proyecto. 


%%%%%%%%%%%%%



Proyecto de Fin de Grado de la especialidad de Computación. Se ha implementado la aplicación para el seguimiento de ejercicios en el aula llamada \textbf{exerClick}. Es una aplicación multiplataforma para móviles, evaluada en Android y en iOS y adaptada a esos sistemas gracias a la plataforma Cordova \cite{apachecordova}. En su implementación se han utilizado las tecnologías web HTML5, CSS3 y Javascript, además de PHP para el servidor.\\

La aplicación permite que los profesores añadan ejercicios y los alumnos le envíen \textit{feedback} sobre su realización mediante dos opciones: marcar una duda en el ejercicio o marcar el ejercicio como finalizado.\\

%%%%%%%%%%%%%
En el presente trabajo de fin de máster desarrollado en la empresa Datik se ha abordado la problemática del procesamiento masivo de datos con el objetivo de dilucidar si es posible hacer frente a este reto con los medio disponibles en la empresa.

Tras analizar las diferentes tecnologías existentes y los recursos disponibles en la empresa se ha construido un cluster altamente escalable basado en máquinas virtuales. Gracias a Apache Cassandra y Apache Spark, dicho cluster es capaz de paralelizar el computo repartiendo las tareas entre los diferentes nodos que lo componen. 

Varios test han sido diseñados y ejecutados tanto en el cluster como en una base de datos MySQL con la intención de verificar el rendimiento ofrecido por el primero. Siendo los resultados obtenidos desfavorables para la infraestructura propuesta, se ha analizado las causas de este hecho. 

Como resultado del análisis, se puede concluir que las tecnologías Apache Cassandra y Apache Spark cumplen con sus propósitos pero los recursos que dispone Datik, a día de hoy, no son los adecuados para sacar partido de estos. Realizando una inversión moderada sería factible apostar por estas tecnologías para lo que se recomienda invertir en contratar un servicio IaaS.

Keywords: ITS, Apache Spark, Apache Cassandra, procesamiento distribuido, Big Data.
