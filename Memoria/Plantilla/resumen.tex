\section*{Resumen}

Proyecto Final del Máster en Sistemas Informáticos Avanzados. Estudio de carácter empírico realizado sobre el rendimiento ofrecido por varias tecnologías emergentes en el campo del Big Data en comparación a una base de datos tradicional a la hora de operar en escenarios que requieren un almacenamiento y procesamiento eficaz de volúmenes masivos de datos.\\

Para llevar a cabo el experimento, dos entornos de prueba totalmente aislados han sido erigidos sobre la misma máquina física. En el primero, se ha configurado un clúster compuesto por cuatro nodos virtuales que operan dentro de una red privada. Dichos nodos han sido dotados de tecnología necesaria para el funcionamiento de Apache Cassandra \cite{apachecassandra} y Apache Spark \cite{apachespark}. En el segundo, se ha instalado una base de datos MySQL tradicional sobre un único nodo virtual que hereda la potencia total de la máquina física. Una vez habiendo poblado las bases de datos mediante un data-set público de aproximadamente 25GB y diseñado unas consultas acorde a la naturaleza de los datos, se han ejecutado dichas consultas para así cuantificar el tiempo de respuesta que necesitan en cada escenario.\\

El estudio evidencia que a la hora de trabajar con volúmenes masivos de datos el binomio entre Apache Cassandra y Apache Spark mejora sustancialmente los tiempos de procesado obtenidos con MySQL además de ofrecer una solución totalmente escalable. No obstante, para gozar de las ventajas que ofrecen estas nuevas tecnologías, se antoja necesario un análisis previo de los datos a tratar, aspecto en el que MySQL ofrece una mayor libertad.\\


Palabras Clave: Big Data, Apache Cassandra, Apache Spark, MySQL, Comparativa.\\
