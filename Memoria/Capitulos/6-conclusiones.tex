%----------------------------------------------------%
%                     CONCLUSIONES                   %
%----------------------------------------------------%

\pagestyle{fancy}

\chapter{Conclusiones y trabajos futuros}
\label{conclusiones}



\section{Conclusiones}

Tal y como se ha podido apreciar durante el transcurso de la presente tesina, muchas son las ventajas ofrecidas por las tecnologías Apache Cassandra y Apache Spark. Entre ellas, cabe resaltar la posibilidad de operar sin punto único de fallo y escalar tanto en almacenamiento como procesamiento superando, además, inequívocamente, los tiempos de respuesta ofrecidos por MySQL.\\

Entramado

por un lado el cambio de paradigma que supone

-Añadir la experiencia adquirida durante este ultimo año y medio trabajando en producción

- Dificil de crear nuevas consultas por que necesita un gran desarrollo, en entornos de constante evolucion pues malamente...

-Cassandra y Spark son tecnologias muy jovenes con un futuro enorme,constante evolucion..., pero en produccion... lo cual hacen que la implantación sea lenta y lleno de baches

- se necesita formar a mucho personal

- se abre un nevo horizonte

- Falta de herramiesntas para gestionar

\section{Trabajos futuros}

//
Calcular los tiempos de insercion correctos aplicando los cambios pertinentes en el disco (ha desvirtuado los resultados obtenidos lo cual es una pena)

// algo sobre spark

//
Se considera de especial interes... 
probar el cluster de mysql añadiendo las restricciones que tiene cassandra y comparar el resultado ofrecido por ambos
40 años a sus espalda tiene mysql, 
igual para una empresa como datik ofrece mas seguridad y estabilidad esta otra solucion (por probar que no quede)

